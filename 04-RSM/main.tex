\documentclass[a4paper,twoside,final]{article}
%----Eingebundene Bibliotheken-----
\usepackage[ngerman]{babel}         % Deutsches Sprachpaket
\usepackage[utf8]{inputenc}         % Eingaben codieren
\usepackage[T1]{fontenc}            % Umlaute codieren, Silbentrennung
\usepackage{amsmath, amssymb}       % Mathe
\usepackage{amsthm,amstext,amsxtra} % Symbole für Mathe
\usepackage{mathtools}              % \Aboxed Boxen in align
\usepackage{wrapfig}                % Bilder umfließen
\usepackage{svg}                    % Vektorgraphiken einbinden
\usepackage{geometry}               % Papierformat
\usepackage{tabularx}               % Tabellen
\usepackage{xcolor,colortbl}        % Farben
\usepackage{graphicx}               % Für Limes Definition wichtig
\usepackage{soul}                   % Unterstreichungen
\usepackage[section]{placeins}      % \Floatbarrier
\usepackage{wrapfig}                % Bilder umfließen
\usepackage{enumerate}              % Aufzählungen
\usepackage{footnote}               % Fußzeilen
\usepackage{booktabs}               % publication quality tables
\usepackage[hyphens]{url}           % \url{}
\usepackage{bm}                     % bold symbols \bm{r}
\usepackage{dsfont}                 % identity matrix \mathds{1}
\usepackage{enumitem}               % itemize Umgebungen customizen
\usepackage{esint}                  % Doppelintegrale
\usepackage{fancyhdr}               % schöne Kopf- und Fußzeilen
\usepackage{lmodern}
\usepackage[labelfont=bf]{subcaption}
\usepackage[square,numbers,sort&compress]{natbib}
\usepackage{mhchem}                 % Chemistry Package
\usepackage[detect-all,
            locale=DE,binary-units,
            exponent-product=\cdot
            ]{siunitx}              % \SI{12}{\gram}
%siunitx stellt für Tabellen den Spaltentyp S bereit ==> Ausrichtung an Dezimaltrennzeichen
\usepackage[position=below,
            tableposition=top,
            format=hang,
            labelfont=it,
            labelfont=bf
            ]{caption}              % Settings für Captions
\usepackage[europeanvoltages,
            europeancurrents,
            europeanresistors,
            americaninductors,
            europeanports
            ]{circuitikz}           % Schaltungen
\usepackage{chngcntr}               % vor hyperref laden!
  \counterwithin*{equation}{section}
  \counterwithin*{figure}{section}
  \counterwithin*{table}{section}

\usepackage[final,
            pdfauthor={Martin Beyer, Vanessa Huth},
            pdfsubject={Fortgeschrittenen-Praktikum},
            pdffitwindow=true,      % resize document window
            pdftitle={Fortgeschrittenen-Praktikum},
            bookmarks=true,         % lesezeichen-Liste
            bookmarksopen=true,     % Lesezeichen geöffnet
            bookmarksopenlevel=1,
            bookmarksnumbered=true,
            colorlinks=true,        % fuer Druckversion auf "false"
            linkcolor=blue,         % Table of Contents, Footnotes
            urlcolor=blue,          % fuer eingebunden URLs
            citecolor=blue,         % Equations, References
            filecolor=blue,
            pdfborder={0 0 0},      % keine Rahmen um Links: {0 0 0}
            ]{hyperref}


% Commands
\renewcommand{\sfdefault}{lmss}     % latin modern sans serif
\newcommand{\R}{\mathbb{R}}         % Reelle Zahlen
\newcommand{\N}{\mathbb{N}}         % Natürliche Zahlen
\newcommand{\C}{\mathbb{C}}         % Komplexe Zahlen
\newcommand{\de}{\mathrm{\;d}}      % Differential
\newcommand{\entspricht}{\mathrel{\widehat{=}}}

\DeclareSIUnit{\eV}{\text{eV}}
\DeclareSIUnit{\voltpeakpeak}{\volt{\textsubscript{pp}}}

% Dokumenteneinstellungen
\setlength{\parindent}{0px}         % remove indent in new paragraph
\setlength{\parindent}{0px}         % keine Absätze durch Leerzeilen im Code
\emergencystretch=1em % Definiert den Leerraum, der innerhalb einer Zeile zusätzlich verteilt werden darf.
\setlength{\topmargin}{-5mm} % 210mm = 8.2677165in
\newlength{\mylength}
\setlength{\mylength}{\paperwidth}
\addtolength{\mylength}{-2in} % standardmäßig wird den Seitenrändern jeweils noch 1in = 25.4mm hinzuaddiert
\setlength{\textwidth}{145mm}
\setlength{\textheight}{230mm}
\addtolength{\mylength}{-\textwidth}
\setlength{\oddsidemargin}{10mm}
\addtolength{\mylength}{-\oddsidemargin}
\setlength{\evensidemargin}{\mylength}
\setlength{\marginparwidth}{1.7cm}
\interfootnotelinepenalty=10000

% Umdefinition von \textcolor ********************************************************
\makeatletter
\renewcommand*{\@textcolor}[3]{%
	\protect\leavevmode
	\begingroup
	\color#1{#2}#3%
	\endgroup
}
\makeatother
% Damit das auch im Mathemodus anwendbar ist und dort z.B. die Leerzeichen nicht wie im Textmodus gesetzt werden.

\begin{document}
\setlength{\marginparsep}{2em}
\renewcommand{\theequation}{\arabic{section}.\arabic{equation}}
\renewcommand{\thefigure}{\arabic{section}.\arabic{figure}}
\renewcommand{\thetable}{\arabic{section}.\arabic{table}}

% Anfang ********************************************************
\begin{center}
\thispagestyle{empty}
  \includegraphics[width=0.75\textwidth]{../UniJena_BildWortMarke_black.pdf}\\[4em]
  \Large
  Ausarbeitung zum Versuch\\[2em]
  \Huge
  Mößbauer Spektroskopie\\
  \vspace{2cm}
  \Large
  Martin Beyer und Vanessa Huth\\[2em]
  Abgabe: 14. Januar 2020\\[2em]
  Betreuer: Dr. Udo Reislöhner\\[5em]
  \begin{flushleft}
  	Bewertung und Ausarbeitung:\\[2em]
		Protokollführung und Form:\\[1em]
		Ergebnisse, Auswertung und Interpretation:\\[1em]
		Bemerkungen und Hinweise des Betreuers:
  \end{flushleft}
\end{center}
\clearpage

\pagestyle{fancy}
\renewcommand{\headrulewidth}{0pt}
\renewcommand{\footrulewidth}{0.5pt}
\renewcommand{\sectionmark}[1]{\markright{#1}}
\fancyhead[RO,LE]{\textbf{Mößbauer Spektroskopie}}
\fancyhead[RE,LO]{\rightmark}
\fancyfoot[LE,RO]{\bfseries\thepage}
\fancyfoot[CO,CE]{Protokoll}
\renewcommand{\headrulewidth}{0.5pt}
\renewcommand{\footrulewidth}{0.5pt}

\setcounter{equation}{0}
\setcounter{figure}{0}

% *********************************************
% ***** KAPITEL 1 *****************************
% *********************************************
\tableofcontents
% \pagenumbering{gobble}% remove page numbering
\newpage
% \pagenumbering{arabic}
\section{Aufgabenstellung} \label{sec:Aufgabenstellung}
%%%%%%%%%%%%%%%%%%%%%%%%%%%%
% *********************************************
% ***** KAPITEL 2 *****************************
% *********************************************
\newpage
\section{Grundlagen} \label{sec:Grundlagen}
\subsection{STM}
\subsection{Rasterkraftmiskroskop}
Das Kernstück des Rasterkraftmikroskop (RKM, englisch auch Atomic Force Microscope (AFM)) ist im Wesentlichen identisch zu dem des Rastertunnelmiskroskops. Auch hier wird eine feine Spitze verwendet, welche in einem charakteristisch kleinem Abstand gegenüber der Probe über diese gerastert wird. Die Höhe der Spitze über der Probe wird in beiden Fällen durch piezoelektrische Stellelemente realisiert. \\
Der Unterschied zwischen beiden Miksrokopiearten besteht in der vorrausgesetzen Probeneigenschaften. Während beim Rastertunnelmiskroskop, da hier ein elektrischer Tunnelstrom gemessen wird, ausschließlich leitende Proben verwendet werden können, sind für Messungen mit dem Rasterkraftmikroskop auch Isolatoren geeignet.\\
\paragraph{Aufbau}
Die Besonderheit im Aufbau des RKM liegt darin, dass hier die Spitze auf eine Blattfeder montiert ist. Das Gebilde aus Blattfeder und Spitze wird Cantilever genannt. Je nach Modi wird die Spitzen-Blattfeder-Konstruktion auf unterschiedliche Weise über die Probe gerastert.

\paragraph{unterschiedliche Messmodi}
Beim RKM werden grundsätzlich 3 verschiedene Messmodi unterschieden.
\subparagraph{Kontaktmodus}
Die feine Spitze wird in diesem Messmodus mit Federkraft auf die Probe gedrückt, steht also in direktem mechanischem Kontakt zur Oberfläche. \\
Erfolgt die Messung nach der \textbf{constant height method}, hält die Spitze eine konstante Höhe in z-Richtung. Die Abtastnadel verbiegt sich hierbei entsprechend der Oberfläche. Diese Messmethode findet auch beim RTM Anwendung. \\
Die \textbf{constant force method} ist entsprechend der constant current method beim RTM und unterscheidet sich nur in der gemessenen Größe, welche konstant gehalten wird. Bei der constant force method wird der Cantilever mittels eines piezoelektrischen Steuerelements so über die Probe gerastert, dass die Auslenkung des Cantilevers und damit die Kraft zwischen Spitze und Probe gleich bleibt. Das Auslenkungssignal der Blattfeder wird hierbei als Regelgröße in einen Regelkreis gegeben, der anschließend die Höhe der Spitze anpasst.\\
Vor- und Nachteile dieser Methoden sind identisch zu den äquivalenten Methoden beim RTM \ref{}.
\subparagraph{Nicht-Kontaktmodus}
Beim Nicht-Konktaktmodus berührt die Spitze die Probe nicht. Der Cantilever wird hierbei mit der Schwingungsfrequenz $\omgea_N$ zum Schwingen angeregt. Beim Schweben der Spitze über die Probe kommt es zu anziehenden Kräften zwischen Spitze und Oberfläche. Meist handelt es sich um Van der Waals Kräfte, je nach Spitzenmaterial können aber auch elektrische oder magnetische Kräfte auftreten. Die Kräfte beeinflussen die Resonanzfrequenz der Feder und die darausfolgendde Änderung wird gemessen.

\subparagraph{Tappping Modus}
Auch bei diesem Messmodus wird die Spitze zu Schwingungen angeregt, tippt die Oberfläche aber bei ihren Schwnigungen leicht an.

\subsubsection{Magnetkraftmikroskop}
Mit dem Magnetkraftmikroskop gelingt es die lokale Magnetstärke in der Probe zu untersuchen. Hierbei ist die Abstastnadel mit einem ferromagnetischen Material beschichtet. Die Messung für jede Bildzeile erfolgt in zwei Durchläufen. Im ersten Durchgang wird mittels eines der vorher beschriebenen Messmodi das Höhenprofil aufgenommen. Im zweiten Durchgang wird die Messnadel mit konstantem Abstand über die Oberfläche gerastert.

\subsection{weitere Rastersondenmikroskope}
Neben den oben genannten, gibt es auch weitere  Rastersondenmikroskope, die die Wechselwirkung zwischen Sonde und Probe nutzen und die die Probe Punkt für Punkt abtasten. Das Grundprinzip aller ist dabei gleich des oben beschriebenen, sie unterscheiden sich im Wesentlichen durch die Abstastung anderer Probeneigenschaften. Dazu zählen zum Beispiel das Optische Rasternahfeldmikroskop (engl. Scanning Nearfiel Optical Microscope, kurz SNOM), das Rasterelektronenmikroskop (REM, engl. SEM) sowie das akustische Rasternahfeldmikroskop (ARNM, engl. scanning near-field acoustic microscope (SNAM). 
\subsection{Piezoeffekt und Piezoelektrische Stellelemente}
% *********************************************
% ***** KAPITEL 3 *****************************
% *********************************************
\section{Messapparatur und Versuchsdurchführung} \label{sec:Versuchsdurchführung}
%%%%%%%%%%%%%%%%%%%%%%%%%%%%%%%%%%%%%%
% *********************************************
% ***** KAPITEL 4 *****************************
% *********************************************
\newpage
\section{Ergebnisse und Diskussion}\label{sec:ErgebnisseUndDiskussion}
%%%%%%%%%%%%%%%%%%%%%%%%%%%%%%%%%%%%%%%

% *********************************************
% ***** KAPITEL 5 *****************************
% *********************************************
\section{Zusammenfassung}
%%%%%%%%%%%%%%%%%%%%%%%%%%%%%%%%%%%
% ***** Literaturverzeichnis ******************

\begin{thebibliography}{xxx}
%%%%%%%%%%%%%%%%%%%%%%%%%%%%%%%%%
\end{thebibliography}

\end{document}
