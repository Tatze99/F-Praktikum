\documentclass[a4paper,twoside,final]{article}
%----Eingebundene Bibliotheken-----
\usepackage[ngerman]{babel}         % Deutsches Sprachpaket
\usepackage[utf8]{inputenc}         % Eingaben codieren
\usepackage[T1]{fontenc}            % Umlaute codieren, Silbentrennung
\usepackage{amsmath, amssymb}       % Mathe
\usepackage{amsthm,amstext,amsxtra} % Symbole für Mathe
\usepackage{mathtools}              % \Aboxed Boxen in align
\usepackage{wrapfig}                % Bilder umfließen
\usepackage{svg}                    % Vektorgraphiken einbinden
\usepackage{geometry}               % Papierformat
\usepackage{tabularx}               % Tabellen
\usepackage{xcolor,colortbl}        % Farben
\usepackage{graphicx}               % Für Limes Definition wichtig
\usepackage{soul}                   % Unterstreichungen
\usepackage[section]{placeins}      % \Floatbarrier
\usepackage{wrapfig}                % Bilder umfließen
\usepackage{enumerate}              % Aufzählungen
\usepackage{footnote}               % Fußzeilen
\usepackage{booktabs}               % publication quality tables
\usepackage[hyphens]{url}           % \url{}
\usepackage{bm}                     % bold symbols \bm{r}
\usepackage{dsfont}                 % identity matrix \mathds{1}
\usepackage{enumitem}               % itemize Umgebungen customizen
\usepackage{esint}                  % Doppelintegrale
\usepackage{fancyhdr}               % schöne Kopf- und Fußzeilen
\usepackage{lmodern}
\usepackage[labelfont=bf]{subcaption}
\usepackage[square,numbers,sort&compress]{natbib}
\usepackage{mhchem}                 % Chemistry Package
\usepackage[detect-all,
            locale=DE,binary-units,
            exponent-product=\cdot
            ]{siunitx}              % \SI{12}{\gram}
%siunitx stellt für Tabellen den Spaltentyp S bereit ==> Ausrichtung an Dezimaltrennzeichen
\usepackage[position=below,
            tableposition=top,
            format=hang,
            labelfont=it,
            labelfont=bf
            ]{caption}              % Settings für Captions
\usepackage[europeanvoltages,
            europeancurrents,
            europeanresistors,
            americaninductors,
            europeanports
            ]{circuitikz}           % Schaltungen
\usepackage{chngcntr}               % vor hyperref laden!
  \counterwithin*{equation}{section}
  \counterwithin*{figure}{section}
  \counterwithin*{table}{section}

\usepackage[final,
            pdfauthor={Martin Beyer, Vanessa Huth},
            pdfsubject={Fortgeschrittenen-Praktikum},
            pdffitwindow=true,      % resize document window
            pdftitle={Fortgeschrittenen-Praktikum},
            bookmarks=true,         % lesezeichen-Liste
            bookmarksopen=true,     % Lesezeichen geöffnet
            bookmarksopenlevel=1,
            bookmarksnumbered=true,
            colorlinks=true,        % fuer Druckversion auf "false"
            linkcolor=blue,         % Table of Contents, Footnotes
            urlcolor=blue,          % fuer eingebunden URLs
            citecolor=blue,         % Equations, References
            filecolor=blue,
            pdfborder={0 0 0},      % keine Rahmen um Links: {0 0 0}
            ]{hyperref}


% Commands
\renewcommand{\sfdefault}{lmss}     % latin modern sans serif
\newcommand{\R}{\mathbb{R}}         % Reelle Zahlen
\newcommand{\N}{\mathbb{N}}         % Natürliche Zahlen
\newcommand{\C}{\mathbb{C}}         % Komplexe Zahlen
\newcommand{\de}{\mathrm{\;d}}      % Differential
\newcommand{\entspricht}{\mathrel{\widehat{=}}}

\DeclareSIUnit{\eV}{\text{eV}}
\DeclareSIUnit{\voltpeakpeak}{\volt{\textsubscript{pp}}}

% Dokumenteneinstellungen
\setlength{\parindent}{0px}         % remove indent in new paragraph
\setlength{\parindent}{0px}         % keine Absätze durch Leerzeilen im Code
\emergencystretch=1em % Definiert den Leerraum, der innerhalb einer Zeile zusätzlich verteilt werden darf.
\setlength{\topmargin}{-5mm} % 210mm = 8.2677165in
\newlength{\mylength}
\setlength{\mylength}{\paperwidth}
\addtolength{\mylength}{-2in} % standardmäßig wird den Seitenrändern jeweils noch 1in = 25.4mm hinzuaddiert
\setlength{\textwidth}{145mm}
\setlength{\textheight}{230mm}
\addtolength{\mylength}{-\textwidth}
\setlength{\oddsidemargin}{10mm}
\addtolength{\mylength}{-\oddsidemargin}
\setlength{\evensidemargin}{\mylength}
\setlength{\marginparwidth}{1.7cm}
\interfootnotelinepenalty=10000

% Umdefinition von \textcolor ********************************************************
\makeatletter
\renewcommand*{\@textcolor}[3]{%
	\protect\leavevmode
	\begingroup
	\color#1{#2}#3%
	\endgroup
}
\makeatother
% Damit das auch im Mathemodus anwendbar ist und dort z.B. die Leerzeichen nicht wie im Textmodus gesetzt werden.


\begin{document}
\setlength{\marginparsep}{2em}
\renewcommand{\theequation}{\arabic{section}.\arabic{equation}}
\renewcommand{\thefigure}{\arabic{section}.\arabic{figure}}
\renewcommand{\thetable}{\arabic{section}.\arabic{table}}

% Anfang ********************************************************
\begin{center}
\thispagestyle{empty}
  \includegraphics[width=0.75\textwidth]{UniJena_BildWortMarke_black.pdf}\\[4em]
  \Large
  Ausarbeitung zum Versuch\\[2em]
  \Huge
  Debye-Sherrer-Verfahren\\
  \vspace{2cm}
  \Large
  Martin Beyer und Vanessa Huth\\[2em]
  Abgabe: 05. November 2019\\[2em]
  Betreuer:\\[5em]
  \begin{flushleft}
  	Bewertung und Ausarbeitung:\\[2em]
		Protokollführung und Form:\\[1em]
		Ergebnisse, Auswertung und Interpretation:\\[1em]
		Bemerkungen und Hinweise des Betreuers:
  \end{flushleft}
\end{center}
\clearpage

\pagestyle{fancy}
\renewcommand{\headrulewidth}{0pt}
\renewcommand{\footrulewidth}{0.5pt}
\renewcommand{\sectionmark}[1]{\markright{#1}}
\fancyhead[RO,LE]{\textbf{Debye-Sherrer-Verfahren}}
\fancyhead[RE,LO]{\rightmark}
\fancyfoot[LE,RO]{\bfseries\thepage}
\fancyfoot[CO,CE]{Protokoll}
\renewcommand{\headrulewidth}{0.5pt}
\renewcommand{\footrulewidth}{0.5pt}

\setcounter{equation}{0}
\setcounter{figure}{0}

% *********************************************
% ***** KAPITEL 1 *****************************
% *********************************************

\section{Aufgabenstellung} \label{sec:Aufgabenstellung}
Lorem ipsum dolor sit amet, consetetur sadipscing elitr, sed diam nonumy eirmod tempor invidunt ut labore et dolore magna aliquyam erat, sed diam voluptua. At vero eos et accusam et justo duo dolores et ea rebum. Stet clita kasd gubergren, no sea takimata sanctus est Lorem ipsum dolor sit amet. Lorem ipsum dolor sit amet, consetetur sadipscing elitr, sed diam nonumy eirmod tempor invidunt ut labore et dolore magna aliquyam erat, sed diam voluptua. At vero eos et accusam et justo duo dolores et ea rebum. Stet clita kasd gubergren, no sea takimata sanctus est Lorem ipsum dolor sit amet.


% *********************************************
% ***** KAPITEL 2 *****************************
% *********************************************
\section{Grundlagen} \label{sec:Grundlagen}

% \begin{figure}[htp]
%     \centering
%     \includegraphics[width=0.8\textwidth]{Bilder/Beispielbild.png}
%     \caption{Beschreibung der Abbildung}
%     \label{fig:Abbildung1}
% \end{figure}\\

% ***** Zwei Bilder nebeneinander *****

% \begin{figure}[htp]
%     \centering
%     \begin{subfigure}{0.45\textwidth}
%         \includegraphics[width=\textwidth]{Bilder/Beispielbild.png}
%     \end{subfigure}
%     \begin{subfigure}{0.45\textwidth}
%         \includegraphics[width=\textwidth]{Bilder/Beispielbild.png}
%     \end{subfigure}
%     \caption{Beschreibung}
% \end{figure}

%***** Tabellen *****
\begin{table}[ht]
	\centering
	\caption{Überschrift der Tabelle}
	\label{tab:Tabelle1}
	\begin{tabular}{l c c c}
		\toprule
	    text & text & text & text\\
	 	\midrule
			text & text & text & text\\
			text & text & text & text
	\end{tabular}
\end{table}

% *********************************************
% ***** KAPITEL 3 *****************************
% *********************************************
\section{Versuchsdurchführung} \label{sec:Versuchsdurchführung}
Prinzip-Zeichnungen, Versuchsanordnung, Geräte,Vorgehensweise, Schaltpläne.


% *********************************************
% ***** KAPITEL 4 *****************************
% *********************************************
\section{Ergebnisse und Diskussion}
Tabellen mit gemessenen und berechneten Werten, Abbildungen, Diagramme usw. mit Unterschriften sowie Vorgehensweise, Kommentare, Fehlerrechnung.\\
Diskussion der Messergebnisse, Diagramme, Fehlerquellen usw. aus physikalischer Sicht, Vergleich mit der Literatur und theoretischen Werten usw.

% *********************************************
% ***** KAPITEL 4 *****************************
% *********************************************
\section{Zusammenfassung}
Lorem ipsum dolor sit amet, consetetur sadipscing elitr, sed diam nonumy eirmod tempor invidunt ut labore et dolore magna aliquyam erat, sed diam voluptua. At vero eos et accusam et justo duo dolores et ea rebum. Stet clita kasd gubergren, no sea takimata sanctus est Lorem ipsum dolor sit amet. Lorem ipsum dolor sit amet, consetetur sadipscing elitr, sed diam nonumy eirmod tempor invidunt ut labore et dolore magna aliquyam erat, sed diam voluptua. At vero eos et accusam et justo duo dolores et ea rebum. Stet clita kasd gubergren, no sea takimata sanctus est Lorem ipsum dolor sit amet.

% ***** Literaturverzeichnis ******************

\begin{thebibliography}{xxx}
	\bibitem{Buch1}
	J. D. Jackson: \textit{Classical Electrodynamics}. Wiley, New York, 3rd edn. (1999).
	\bibitem{Buch2}
	M. Kasha: \textit{Molecular Excitons in Small Aggregates}. In: B. DiBartolo (Ed.), \textit{Spectroscopy of the Excited State}, 337–363. Plenum Press, New York (1976).
	\bibitem{Internetquelle}
	Lustiges Video: \url{https://www.youtube.com/watch?v=u8u1Hhhmygo} (Stand: 20.02.2022)
\end{thebibliography}

\end{document}
