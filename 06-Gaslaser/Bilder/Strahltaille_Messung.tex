\begin{figure}[h!]
  \centering
  \begin{tikzpicture}
    \begin{axis}[disabledatascaling, width=\textwidth, height=6cm, ylabel=Strahltaille $z_r$ in \si{\micro\metre}, xlabel=Resonatorlänge $L$ in \si{\metre}, xmin=0, xmax=1, ymin = 0, samples = 200, domain=0:1]
      \addplot[blue, thick] {(2*0.5*x-x^2)^(1/4)*(0.6328/(2*pi))^(1/2)*1000};
      \addplot [only marks]
        plot [error bars/.cd, y dir=both, y explicit]
        table [y error plus=y-max, y error minus=y-min] {\mytable};
    \end{axis}
\end{tikzpicture}
  \caption{Berechnete Strahltaille für verschiedene Resonatorlängen $L$. Für $L=\SI{.75}{\metre}$ wurde kein reeller Wert ermittelt. Es wird der erste Wert gezeigt, welcher im Messbereich einen reellen Taillenradius lieferte.}
  \label{fig:Strahltaille_Theorie}
\end{figure}
