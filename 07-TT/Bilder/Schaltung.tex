\begin{figure}[htp]
  \centering
  \begin{tikzpicture}
    \draw (0,0) to[current source] (3,0) to (3,3) to[R=$R$] (0,3) to (0,0);
    \draw (0,1.5)node[circ]{} to (1.5,1.5)node[draw,circle,fill=white] {V} to (3,1.5)node[circ]{};
    \node (A) at (1.5,-.7){Zweileiterschaltung};
    \begin{scope}[shift={(6,0)}]
      \draw (0,0) to[current source] (3,0) to (3,3) to[R=$R$] (0,3) to (0,0);
      \draw (0.7,3)node[circ]{} to(0.7,1.5) to (1.5,1.5)node[draw,circle,fill=white] {V} to (2.3,1.5) to (2.3,3)node[circ]{};
      \node (A) at (1.5,-.7){Vierleiterschaltung};
    \end{scope}
  \end{tikzpicture}
  \caption{Zwei- und Vierleiterschaltung im Vergleich. Da die Abstände von den Messgeräten zu den Temperaturwiderständen in der Regel groß sind, sind Leitungswiderstände nicht vernachlässigbar. Bei der Vierleiterschaltung wird ein zusätzliches Leitungspaar verlegt, welches die Spannung kurz vor und hinter dem Widerstand abgreift. Aufgrund des viel größeren Widerstandes des Voltmeters gegenüber der Stromquelle, fließt in diesem Stromkreis fast kein Strom, was nahezu kein Spannungsabfall über der Leitung ergibt.}
  \label{fig:Schaltung}
\end{figure}
