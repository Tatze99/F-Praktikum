\begin{figure}[htp]
  \centering
  \begin{tikzpicture}
    \begin{axis}[width=\textwidth, height=6cm, ylabel=$\ln\qty(\frac{R_0}{R})$, xmin=0.0035, xmax=0.0115, xlabel={$\frac{1}{T}$ in \si{\per\kelvin}}, legend cell align={left}, xtick={0.004, 0.006, 0.008, 0.010}, xticklabels={0.004, 0.006, 0.008, 0.010},  xtick scale label code/.code={}]
      \addplot[blue, thick] file[x index=1,y index=2]{Daten/Halbleitermessung_Logarithmisch_Diode.txt};
      \addplot[orange, thick, domain=0.0035:0.0115]{2.62583-1698.34845*x};
      \legend{Messdaten, Lineare Anpassung};
      \node[anchor=west] (A) at (0.0037,-14){\begin{tabular}{|c | c|}
        \hline
        Fit & $y = a + b\cdot \frac{1}{T}$\\
        \hline
        $a$ & $2.63\pm 0.012$ \\
        $b$ [\si{\kelvin}] & $-1698.3\pm1.3$\\
        $R^2$ & 0.99982\\
        \hline
      \end{tabular}};
    \end{axis}
  \end{tikzpicture}
  \caption{Natürlicher Logarithmus des inversen Widerstandes über der inversen Temperatur mit $R_0 = \SI{1}{\ohm}$. Im Bereich tieferer Temperaturen \SIrange{80}{150}{\kelvin} wurde eine lineare Anpassung durchgeführt.}
  \label{fig:Halbleitermessung}
\end{figure}
