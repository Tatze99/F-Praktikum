\begin{figure}[htp]
  \centering
  \begin{tikzpicture}
    \begin{axis}[width=0.98\textwidth, height=6cm, ylabel=Spannung $U$ in \si{\volt}, xlabel=Wellenlänge $\lambda$ in \si{\micro\metre},xmin=1500, xmax=14000, legend cell align={left}, scaled x ticks={real:1000}, xtick scale label code/.code={}]
      \addplot[blue, thick] table[x index=0,y index=1, col sep=semicolon]{Daten/Reflektivitaet_SiO2.csv};
      \addlegendentry{\ce{SiO2}}
      \addplot[orange, thick] table[x index=0,y index=2, col sep=semicolon]{Daten/Reflektivitaet_SiO2.csv};
      \addlegendentry{Referenz (Spiegel)}
      \addplot[green!50!black, thick] table[x index=0,y index=3, col sep=semicolon]{Daten/Reflektivitaet_SiO2.csv};
      \addlegendentry{Reflexionsrate}
    \end{axis}
  \end{tikzpicture}
  \caption{Messungen des Signals nach Reflexion an \ce{SiO2} Plättchen bzw. an Aluminiumspiegel. Die Darstellung ist nicht maßstabsgetreu. Zusätzlich ist die Reflexionsrate, also das Verhältnis aus Signal zu Referenz, dargestellt. Der zweite Peak des \ce{SiO2} Signal entspricht ungefähr \SI{50}{\percent} des ursprünglichen Signals.}
  \label{fig:RefliktivitätSiO2}
\end{figure}

% \begin{figure}[htp]
%   \centering
%     \includegraphics[width=\textwidth]{Bilder/Reflektivität_SiO2.pdf}
%     \caption{Messungen des Signals nach Reflexion an \ce{SiO2} Plättchen bzw. an Aluminiumspiegel. Die Darstellung ist nicht maßstabsgetreuen. Zusätzlich ist die Reflexionsrate, also das Verhältnis aus Signal zu Referenz, dargestellt. Der zweite Peak des \ce{SiO2} Signal entspricht ungefähr \SI{50}{\percent} des ursprünglichen Signals. }
% \end{figure}
