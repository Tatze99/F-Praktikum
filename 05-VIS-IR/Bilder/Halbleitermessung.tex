\begin{figure}[htp]
  \centering
  \begin{tikzpicture}
    \begin{axis}[axis on top, disabledatascaling, width=\textwidth, height=6cm, ylabel=Absorptionsrate, xlabel=Wellenlänge $\lambda$ in \si{\nano\metre}, xmin = 950, xmax=2150, xtick={}, ymin =-.1, ymax = 1.15, legend cell align={left}]
      \filldraw[draw=white!80!orange,fill=white!90!orange] (1020,-.1) rectangle (1080,1.2);
      \draw[thick, white!50!orange] (1050,-.1) -- (1050,1.2);
      \filldraw[draw=green!50!black, draw opacity = 0.2,fill=green!50!black, fill opacity = 0.1] (1770,-.1) rectangle (1830,1.2);
      \draw[thick, green!50!black, opacity = 0.5] (1800,-.1) -- (1800,1.2);
      \addplot[orange, thick] file[x index=1,y index=2]{Daten/Absorption_Smooth_Kalibrierung_Glasprisma_Wolframlampe5V_kleinesHalbleiterplaettchen_3_6Grad_500nm_all100nmMaker_Spalt_500mu.txt};
      \addlegendentry{Silizium}
      \addplot[green!50!black, thick] file[x index=1,y index=2]{Daten/Absorption_Smooth_Kalibrierung_Glasprisma_Wolframlampe5V_mitgrossemHalbleiterplaettchen_3_6Grad_500-3200nm_mitMarkern2_Spalt_500mu.txt};
      \addlegendentry{Germanium}
    \end{axis}
  \end{tikzpicture}
  \caption{Transmissionsrate von Silizium und Germanium.}
  \label{fig:Halbleitermessung_relativ}
\end{figure}
