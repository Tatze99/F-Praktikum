\begin{figure}[htp]
  \centering
  \begin{tikzpicture}
    \begin{axis}[width=0.98\textwidth, height=6cm, ylabel=Energiedichte in a.u., xlabel=Wellenlänge $\lambda$ in \si{\micro\metre},xmin=500, xmax=13000, legend cell align={left}, scaled x ticks={real:1000}, xtick scale label code/.code={}]
      \addplot[blue, thick] file[x index=1,y index=2]{Daten/Kalibrierung_Wolframlampe500-3200nm.txt};
      \addlegendentry{Wolframlampe \ce{SiO2}}
      \addplot[orange, thick] file[x index=1,y index=2]{Daten/Kalibrierung_NaClPrisma_Keramikstrahler_19_4V_3_6Grad_600nm_bis11_5mu_all200nmMaker_Spalt_1500mu_gut.txt};
      \addlegendentry{Keramikstrahler NaCl}
      \addplot[green!50!black, thick] file[x index=1,y index=2]{Daten/Kalibrierung_KBrPrisma_Keramik19_4V_3_6Grad_1500nm_all500nmMaker_Spalt_1500mu.txt};
      \addlegendentry{Keramikstrahler KBr}
    \end{axis}
  \end{tikzpicture}
  \caption{Referenzspektren der Wolframlampe (blau) und des Keramikstrahlers (orange, grün) unter Verwendung verschiedener Prismen. Das Maximum der Strahlungsleistung liegt für den Keramikstrahler bei höheren Wellenlängen.}
  \label{fig:Referenzspektren}
\end{figure}
